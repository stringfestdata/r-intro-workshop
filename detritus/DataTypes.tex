\documentclass[ignorenonframetext,]{beamer}
\setbeamertemplate{caption}[numbered]
\setbeamertemplate{caption label separator}{: }
\setbeamercolor{caption name}{fg=normal text.fg}
\beamertemplatenavigationsymbolsempty
\usepackage{lmodern}
\usepackage{amssymb,amsmath}
\usepackage{ifxetex,ifluatex}
\usepackage{fixltx2e} % provides \textsubscript
\ifnum 0\ifxetex 1\fi\ifluatex 1\fi=0 % if pdftex
  \usepackage[T1]{fontenc}
  \usepackage[utf8]{inputenc}
\else % if luatex or xelatex
  \ifxetex
    \usepackage{mathspec}
  \else
    \usepackage{fontspec}
  \fi
  \defaultfontfeatures{Ligatures=TeX,Scale=MatchLowercase}
\fi
% use upquote if available, for straight quotes in verbatim environments
\IfFileExists{upquote.sty}{\usepackage{upquote}}{}
% use microtype if available
\IfFileExists{microtype.sty}{%
\usepackage{microtype}
\UseMicrotypeSet[protrusion]{basicmath} % disable protrusion for tt fonts
}{}
\newif\ifbibliography
\hypersetup{
            pdftitle={Data Types in R},
            pdfborder={0 0 0},
            breaklinks=true}
\urlstyle{same}  % don't use monospace font for urls
\usepackage{color}
\usepackage{fancyvrb}
\newcommand{\VerbBar}{|}
\newcommand{\VERB}{\Verb[commandchars=\\\{\}]}
\DefineVerbatimEnvironment{Highlighting}{Verbatim}{commandchars=\\\{\}}
% Add ',fontsize=\small' for more characters per line
\usepackage{framed}
\definecolor{shadecolor}{RGB}{248,248,248}
\newenvironment{Shaded}{\begin{snugshade}}{\end{snugshade}}
\newcommand{\KeywordTok}[1]{\textcolor[rgb]{0.13,0.29,0.53}{\textbf{#1}}}
\newcommand{\DataTypeTok}[1]{\textcolor[rgb]{0.13,0.29,0.53}{#1}}
\newcommand{\DecValTok}[1]{\textcolor[rgb]{0.00,0.00,0.81}{#1}}
\newcommand{\BaseNTok}[1]{\textcolor[rgb]{0.00,0.00,0.81}{#1}}
\newcommand{\FloatTok}[1]{\textcolor[rgb]{0.00,0.00,0.81}{#1}}
\newcommand{\ConstantTok}[1]{\textcolor[rgb]{0.00,0.00,0.00}{#1}}
\newcommand{\CharTok}[1]{\textcolor[rgb]{0.31,0.60,0.02}{#1}}
\newcommand{\SpecialCharTok}[1]{\textcolor[rgb]{0.00,0.00,0.00}{#1}}
\newcommand{\StringTok}[1]{\textcolor[rgb]{0.31,0.60,0.02}{#1}}
\newcommand{\VerbatimStringTok}[1]{\textcolor[rgb]{0.31,0.60,0.02}{#1}}
\newcommand{\SpecialStringTok}[1]{\textcolor[rgb]{0.31,0.60,0.02}{#1}}
\newcommand{\ImportTok}[1]{#1}
\newcommand{\CommentTok}[1]{\textcolor[rgb]{0.56,0.35,0.01}{\textit{#1}}}
\newcommand{\DocumentationTok}[1]{\textcolor[rgb]{0.56,0.35,0.01}{\textbf{\textit{#1}}}}
\newcommand{\AnnotationTok}[1]{\textcolor[rgb]{0.56,0.35,0.01}{\textbf{\textit{#1}}}}
\newcommand{\CommentVarTok}[1]{\textcolor[rgb]{0.56,0.35,0.01}{\textbf{\textit{#1}}}}
\newcommand{\OtherTok}[1]{\textcolor[rgb]{0.56,0.35,0.01}{#1}}
\newcommand{\FunctionTok}[1]{\textcolor[rgb]{0.00,0.00,0.00}{#1}}
\newcommand{\VariableTok}[1]{\textcolor[rgb]{0.00,0.00,0.00}{#1}}
\newcommand{\ControlFlowTok}[1]{\textcolor[rgb]{0.13,0.29,0.53}{\textbf{#1}}}
\newcommand{\OperatorTok}[1]{\textcolor[rgb]{0.81,0.36,0.00}{\textbf{#1}}}
\newcommand{\BuiltInTok}[1]{#1}
\newcommand{\ExtensionTok}[1]{#1}
\newcommand{\PreprocessorTok}[1]{\textcolor[rgb]{0.56,0.35,0.01}{\textit{#1}}}
\newcommand{\AttributeTok}[1]{\textcolor[rgb]{0.77,0.63,0.00}{#1}}
\newcommand{\RegionMarkerTok}[1]{#1}
\newcommand{\InformationTok}[1]{\textcolor[rgb]{0.56,0.35,0.01}{\textbf{\textit{#1}}}}
\newcommand{\WarningTok}[1]{\textcolor[rgb]{0.56,0.35,0.01}{\textbf{\textit{#1}}}}
\newcommand{\AlertTok}[1]{\textcolor[rgb]{0.94,0.16,0.16}{#1}}
\newcommand{\ErrorTok}[1]{\textcolor[rgb]{0.64,0.00,0.00}{\textbf{#1}}}
\newcommand{\NormalTok}[1]{#1}

% Prevent slide breaks in the middle of a paragraph:
\widowpenalties 1 10000
\raggedbottom

\AtBeginPart{
  \let\insertpartnumber\relax
  \let\partname\relax
  \frame{\partpage}
}
\AtBeginSection{
  \ifbibliography
  \else
    \let\insertsectionnumber\relax
    \let\sectionname\relax
    \frame{\sectionpage}
  \fi
}
\AtBeginSubsection{
  \let\insertsubsectionnumber\relax
  \let\subsectionname\relax
  \frame{\subsectionpage}
}

\setlength{\parindent}{0pt}
\setlength{\parskip}{6pt plus 2pt minus 1pt}
\setlength{\emergencystretch}{3em}  % prevent overfull lines
\providecommand{\tightlist}{%
  \setlength{\itemsep}{0pt}\setlength{\parskip}{0pt}}
\setcounter{secnumdepth}{0}

\title{Data Types in R}
\date{}

\begin{document}
\frame{\titlepage}

\begin{frame}[fragile]{Creating objects}

R stores both data and output from data analysis (as well as everything
else) in objects.

Creating a new object is as easy as typing the object's name and
assigning a value to it. There are multiple ways to assign values to
objects in R.

As in many computer languages you can use the equal sign (=) as an
assignment operator. Copy the following code into your script, and run
it.

\begin{Shaded}
\begin{Highlighting}[]
\CommentTok{#assign the value of the square root of 100 to }
\CommentTok{#the object, variable 'result'}
\NormalTok{result=}\KeywordTok{sqrt}\NormalTok{(}\DecValTok{100}\NormalTok{)}
\end{Highlighting}
\end{Shaded}

\end{frame}

\begin{frame}[fragile]

You are more likely, however, to see \texttt{\textless{}-} used to
assign values to objects:

\begin{Shaded}
\begin{Highlighting}[]
\CommentTok{#assign the value of the square root of 100 to}
\CommentTok{#the object, variable 'result'}
\NormalTok{result<-}\KeywordTok{sqrt}\NormalTok{(}\DecValTok{100}\NormalTok{)}
\end{Highlighting}
\end{Shaded}

\end{frame}

\begin{frame}[fragile]{Printing objects}

Now we have created an object called result. It currently has the value
of 10. To verify that, we can type the name of the object, and R will
print the value stored for it:

\begin{Shaded}
\begin{Highlighting}[]
\NormalTok{result}
\NormalTok{## [1] 10}
\end{Highlighting}
\end{Shaded}

You can think of an object in R as a shoebox (object) with a name. Here,
the name of the shoebox is ``result.'' Imagine that it is holding a
scrap of paper with the number 10 on it.

\end{frame}

\begin{frame}[fragile]{Naming objects}

You can name your object anything you want, with a few exceptions. You
can store any value in it, with a few exceptions.

For instance we could have named it \texttt{ScoobyDoo} if we wanted, but
that's not a very informative name.

\end{frame}

\section{EVERYTHING IN R IS AN
OBJECT.}\label{everything-in-r-is-an-object.}

\begin{frame}[fragile]{R has six basic data types:}

\begin{itemize}[<+->]
\tightlist
\item
  character: \texttt{"a"}, \texttt{"swc"}
\item
  numeric: \texttt{2}, \texttt{15.5}
\item
  integer: \texttt{2L} (the \texttt{L} tells R to store this as an
  integer)
\item
  logical: \texttt{TRUE}, \texttt{FALSE}
\item
  complex: \texttt{1+4i} (complex numbers with real and imaginary parts)
\end{itemize}

\end{frame}

\begin{frame}[fragile]{Inspecting an object's data type}

R provides many functions to examine features of objects. Two important
ones:

\begin{itemize}[<+->]
\tightlist
\item
  \texttt{class()} - what kind of object is it (high-level)?
\item
  \texttt{length()} - how long is it?
\end{itemize}

For example, we can get some information about the \texttt{result}
object that we just created:

\begin{Shaded}
\begin{Highlighting}[]
\CommentTok{#find class of object 'result'}
\KeywordTok{class}\NormalTok{(result)}
\NormalTok{## [1] "numeric"}

\CommentTok{#find length of object 'result'}
\KeywordTok{length}\NormalTok{(result)}
\NormalTok{## [1] 1}
\end{Highlighting}
\end{Shaded}

\end{frame}

\begin{frame}{Data Structures in R}

There are several different data structures in R. If you take a longer
introduction to R you'll learn about arrays, lists, matrices, etc.

We will skip to the two most common data types: vectors and data frames.

\end{frame}

\begin{frame}[fragile]{Vectors}

Guess what, \emph{you've been using them already!}

Vectors are the building block of R. Simply, they are one-dimensional
collections of values of the same type (integer, character, logical,
etc.)

\begin{block}{Let's build a few vectors from scratch using the
\texttt{c()} function\ldots{}}

\begin{Shaded}
\begin{Highlighting}[]
\CommentTok{#A vector of numbers}
\NormalTok{x<-}\KeywordTok{c}\NormalTok{(}\DecValTok{5}\NormalTok{,}\DecValTok{9}\NormalTok{,}\DecValTok{11}\NormalTok{)}
\NormalTok{x}
\NormalTok{## [1]  5  9 11}
\end{Highlighting}
\end{Shaded}

\end{block}

\end{frame}

\begin{frame}[fragile]

\begin{Shaded}
\begin{Highlighting}[]
\CommentTok{#A vector of logical values}
\NormalTok{y<-}\KeywordTok{c}\NormalTok{(}\OtherTok{TRUE}\NormalTok{,}\OtherTok{FALSE}\NormalTok{,}\OtherTok{TRUE}\NormalTok{,}\OtherTok{FALSE}\NormalTok{,}\OtherTok{FALSE}\NormalTok{)}
\NormalTok{y}
\NormalTok{## [1]  TRUE FALSE  TRUE FALSE FALSE}
\end{Highlighting}
\end{Shaded}

\begin{Shaded}
\begin{Highlighting}[]
\CommentTok{#A vector of character strings}
\NormalTok{people<-}\KeywordTok{c}\NormalTok{(}\StringTok{"Jack"}\NormalTok{,}\StringTok{"Jill"}\NormalTok{,}\StringTok{"Jim"}\NormalTok{,}\StringTok{"June"}\NormalTok{)}
\NormalTok{people}
\NormalTok{## [1] "Jack" "Jill" "Jim"  "June"}
\end{Highlighting}
\end{Shaded}

\begin{Shaded}
\begin{Highlighting}[]
\CommentTok{#A vector with one value}
\NormalTok{v<-}\KeywordTok{c}\NormalTok{(}\DecValTok{15}\NormalTok{)}
\NormalTok{v}
\NormalTok{## [1] 15}
\end{Highlighting}
\end{Shaded}

\end{frame}

\begin{frame}[fragile]

The elements of a vector must \textbf{all have the same mode, or data
type.}

You can have a vector consisting of three character strings, or three
integer elements, but not a vector with one integer element and two
character string elements.

\begin{Shaded}
\begin{Highlighting}[]
\NormalTok{k<-}\KeywordTok{c}\NormalTok{(}\DecValTok{1}\NormalTok{,}\DecValTok{2}\NormalTok{,}\DecValTok{3}\NormalTok{,}\StringTok{"BOO!"}\NormalTok{)}
\NormalTok{k}
\NormalTok{## [1] "1"    "2"    "3"    "BOO!"}
\end{Highlighting}
\end{Shaded}

\end{frame}

\begin{frame}[fragile]{Accessing values of a vector}

Once your vector is defined, you will want to access values from it. For
example, you may want to get the fourth value in the vector, or the
second to last, or so forth.

The syntax for accessing values is \texttt{vectorname{[}index{]}} where
vectorname is the name of the vector, and index is the index number of
the element you are looking for.

Using the \texttt{people} vector we created earlier, here are some
examples.

\end{frame}

\begin{frame}[fragile]

\begin{Shaded}
\begin{Highlighting}[]
\CommentTok{#print the people vector just for reference}
\NormalTok{people}
\NormalTok{## [1] "Jack" "Jill" "Jim"  "June"}
\end{Highlighting}
\end{Shaded}

\begin{Shaded}
\begin{Highlighting}[]
\CommentTok{#get the first value from the vector}
\NormalTok{people[}\DecValTok{1}\NormalTok{]}
\NormalTok{## [1] "Jack"}
\end{Highlighting}
\end{Shaded}

\begin{Shaded}
\begin{Highlighting}[]
\CommentTok{#R will evaluate any math within the brackets...}
\NormalTok{people[}\DecValTok{1}\OperatorTok{+}\DecValTok{2}\NormalTok{]}
\NormalTok{## [1] "Jim"}
\NormalTok{people[}\KeywordTok{length}\NormalTok{(people)]}
\NormalTok{## [1] "June"}
\end{Highlighting}
\end{Shaded}

\begin{Shaded}
\begin{Highlighting}[]
\CommentTok{#print the 1st and 3rd values in different order}
\NormalTok{people[}\KeywordTok{c}\NormalTok{(}\DecValTok{1}\NormalTok{,}\DecValTok{3}\NormalTok{)]}
\NormalTok{## [1] "Jack" "Jim"}
\NormalTok{people[}\KeywordTok{c}\NormalTok{(}\DecValTok{3}\NormalTok{,}\DecValTok{1}\NormalTok{)]}
\NormalTok{## [1] "Jim"  "Jack"}
\end{Highlighting}
\end{Shaded}

\end{frame}

\begin{frame}{Data Frames}

Datasets for statistical analysis are typically stored in data frames in
R.

Data frames are rectangular, where the columns are variables and the
rows are observations of those variables. This is very similar to a
table in Excel or in a database.

Columns of data frames are\ldots{} \textbf{vectors}!

And, like vectors, data frame columns can be of several different data
types, but all entries in the same vector must be the same type.

Also, all columns in a data frame must be of the same equal length.

\end{frame}

\begin{frame}[fragile]{Creating with data.frame()}

Data frames can be manually created with \texttt{data.frame()}.

The elements of a data frame are almost always named.

\begin{Shaded}
\begin{Highlighting}[]
\NormalTok{mydata <-}\StringTok{ }\KeywordTok{data.frame}\NormalTok{(}\DataTypeTok{diabetic =} \KeywordTok{c}\NormalTok{(}\OtherTok{TRUE}\NormalTok{, }\OtherTok{FALSE}\NormalTok{, }\OtherTok{TRUE}\NormalTok{, }\OtherTok{FALSE}\NormalTok{), }
                     \DataTypeTok{height =} \KeywordTok{c}\NormalTok{(}\DecValTok{65}\NormalTok{, }\DecValTok{69}\NormalTok{, }\DecValTok{71}\NormalTok{, }\DecValTok{73}\NormalTok{))}
\NormalTok{mydata}
\NormalTok{##   diabetic height}
\NormalTok{## 1     TRUE     65}
\NormalTok{## 2    FALSE     69}
\NormalTok{## 3     TRUE     71}
\NormalTok{## 4    FALSE     73}
\end{Highlighting}
\end{Shaded}

\end{frame}

\begin{frame}[fragile]{Subsetting dataframes}

Data frames, unlike vectors, are two-dimensional structures. We subset
them using the formula {[}rows, columns{]}.

\begin{Shaded}
\begin{Highlighting}[]
\CommentTok{# row 3 column 2}
\NormalTok{mydata[}\DecValTok{3}\NormalTok{,}\DecValTok{2}\NormalTok{]}
\NormalTok{## [1] 71}
\end{Highlighting}
\end{Shaded}

\begin{Shaded}
\begin{Highlighting}[]
\CommentTok{# using column name}
\NormalTok{mydata[}\DecValTok{1}\OperatorTok{:}\DecValTok{2}\NormalTok{, }\StringTok{"height"}\NormalTok{]}
\NormalTok{## [1] 65 69}
\end{Highlighting}
\end{Shaded}

\begin{Shaded}
\begin{Highlighting}[]
\CommentTok{# all rows of column "height"}
\NormalTok{mydata[,}\StringTok{"diabetic"}\NormalTok{]}
\NormalTok{## [1]  TRUE FALSE  TRUE FALSE}
\end{Highlighting}
\end{Shaded}

\end{frame}

\begin{frame}[fragile]{Subsetting entire columns}

To subset an entire column of a dataframe, we use \$ followed by that
column name.

\begin{Shaded}
\begin{Highlighting}[]
\CommentTok{# subsetting creates a numeric vector}
\NormalTok{mydata}\OperatorTok{$}\NormalTok{height}
\NormalTok{## [1] 65 69 71 73}
\end{Highlighting}
\end{Shaded}

This function will be \emph{extremely important} when we come to do
operations on the data.

\end{frame}

\begin{frame}[fragile]{Naming dataframe columns}

\texttt{colnames(data\_frame)} returns the column names of the data
frame.

\texttt{colnames(data\_frame)\ \textless{}-\ c("some",\ "names")} will
assign column names to the data frame.

\begin{Shaded}
\begin{Highlighting}[]
\CommentTok{# get column names}
\KeywordTok{colnames}\NormalTok{(mydata)}
\NormalTok{## [1] "diabetic" "height"}
\end{Highlighting}
\end{Shaded}

\begin{Shaded}
\begin{Highlighting}[]
\CommentTok{# assign column names}
\KeywordTok{colnames}\NormalTok{(mydata) <-}\StringTok{ }\KeywordTok{c}\NormalTok{(}\StringTok{"Diabetic"}\NormalTok{, }\StringTok{"Height"}\NormalTok{)}
\KeywordTok{colnames}\NormalTok{(mydata)}
\NormalTok{## [1] "Diabetic" "Height"}
\end{Highlighting}
\end{Shaded}

\end{frame}

\begin{frame}[fragile]

To change one column name, just use indexing.

\begin{Shaded}
\begin{Highlighting}[]
\KeywordTok{colnames}\NormalTok{(mydata)[}\DecValTok{1}\NormalTok{] <-}\StringTok{ "Diabetes"}
\KeywordTok{colnames}\NormalTok{(mydata)}
\NormalTok{## [1] "Diabetes" "Height"}
\end{Highlighting}
\end{Shaded}

\end{frame}

\section{Questions?}\label{questions}

\end{document}
